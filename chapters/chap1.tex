
\chapter{Vorstellung der Cloud-Native-Anwendung}
Die in der Laborarbeit entwickelte Cloud-Native-Anwendung Green Grid GGuide hat das Ziel, Nutzern wie Energieunternehmen oder Grundstückseigentümern bei der Suche nach geeigneten Standorten für erneuerbare Energien zu unterstützen. Hierfür werden sowohl historische als auch aktuelle Wetterdaten sowie geographische Gegebenheiten analysiert, um potenzielle Standorte für Wind- oder Solaranlagen zu identifizieren und umfassend zu bewerten.

Die Bewertung dieser Standorte erfolgt anhand verschiedener Kriterien. Neben geeigneten Wetterbedingungen wird auch die Entfernung des Standorts zu bestehenden Stromleitungen berücksichtigt, da dies entscheidend für die wirtschaftliche Rentabilität einer möglichen Anlage ist. Zudem wird geprüft, ob sich der Standort in einem Naturschutzgebiet, im Wald oder in einem Wohngebiet befindet, da diese Faktoren Einfluss auf die Genehmigungschancen für den Bau einer Anlage haben.

Die Anwendung bietet den Nutzern eine intuitive Benutzeroberfläche inform einer interaktiven Karte und einstellbaren Parametern wie den Suchradius zur Visualisierung dieser Informationen, die bei der Entscheidungsfindung unterstützt und somit zur Planung von Wind- und Solaranlagen beiträgt.