\chapter{Vorteile und Nachteile der Cloud-Native-Realisation}

In diesem Kapitel werden die Vorteile und Nachteile der Cloud-Native-Realisierung der Anwendung zur Unterstützung der Standortfindung für erneuerbare Energien diskutiert. Zudem werden mögliche alternative Realisierungsmöglichkeiten aufgezeigt.

\section{Vorteile und Nachteile der Cloud-Native-Realisierung}

Die Realisierung der Anwendung als Cloud-Native-Anwendung bietet mehrere Vorteile:

\textbf{Skalierbarkeit:} Bei steigender Nachfrage, beispielsweise wenn viele Nutzer gleichzeitig potenzielle Standorte für Wind- oder Solaranlagen abfragen, können zusätzliche Instanzen der Microservices schnell bereitgestellt werden, um die Last zu bewältigen. Dies wird durch die Containerisierung und die Orchestrierung mit Kubernetes ermöglicht. Unbenötigte Instanzen können bei geringerer Nachfrage effizient heruntergefahren werden, was Ressourcen spart.

\textbf{Unabhängige Entwicklung von Microservices:} Die Nutzung von Microservices ermöglicht es verschiedenen Teammitgliedern, parallel an spezifischen Komponenten der Anwendung zu arbeiten. So kann ein Teammitglied beispielsweise an der Verarbeitung von Wetterdaten arbeiten, während ein anderes Teammitglied gleichzeitig den Geo-Microservice weiterentwickelt. Dies beschleunigt die Einführung neuer Funktionen, wie die Integration zusätzlicher Wetterdatenquellen, ohne die gesamte Anwendung zu beeinträchtigen.

\textbf{Resilienz und Verfügbarkeit:} Der Einsatz von Kubernetes-Primitiven wie Liveness- und Readiness-Probes gewährleistet, dass fehlerhafte Instanzen automatisch erkannt und neu gestartet werden. Wenn beispielsweise Fehler in der Wetterdatenabfrage in der Datenbank auftreten, führt dies nicht zu einem Ausfall der gesamten Anwendung.

\textbf{Automatisierung in der Bereitstellung und einfacher Betrieb:} Die Implementierung von CI/CD-Pipelines zur Automatisierung des Entwicklungs- und Bereitstellungsprozesses reduziert menschliche Fehler und beschleunigt die Zeit von der Entwicklung bis zur Produktion. Dies ist entscheidend, um zeitnah auf Änderungen in den regulatorischen Anforderungen für erneuerbare Energien zu reagieren und neue Datenquellen schnell zu integrieren. Hier eingesetzte Tools wie Ansible erleichtern das Infrastrukturmanagement und steigern die Effizienz der Anwendung.

Trotz dieser Vorteile gibt es auch einige \textbf{Nachteile:}

\textbf{Erhöhter Overhead durch Containerisierung:} Jeder Microservice läuft in einem eigenen Container, was zusätzliche Ressourcen benötigt und die Infrastrukturverwaltung komplizierter macht.

\textbf{Herausforderungen bei der Netzwerkkommunikation:} Die Sicherheit und Effizienz der Datenübertragung zwischen dem Weather-Microservice und der PostgreSQL-Datenbank können Schwierigkeiten bereiten. Auch die Kommunikation des Frontends mit den Microservices über ein Netzwerk kann potenzielle Latenzen und eine komplexere Architektur mit sich bringen. Wenn beispielsweise der Geo-Microservice aufgrund von Netzwerkproblemen nicht erreichbar ist, kann dies die gesamte Anwendung beeinträchtigen und den Zugriff auf wichtige Informationen für die Standortbewertung verzögern.

\section{Alternative Realisierungsmöglichkeiten}

Alternativen zur Cloud-Native-Architektur sind die monolithische Architektur und On-Premise-Lösungen. Jede dieser Alternativen bietet spezifische Vor- und Nachteile für die vorliegende Anwendung.

\subsection{Monolithische Architektur}

Eine monolithische Architektur integriert alle Funktionen der Anwendung in einer einzigen, großen Codebasis. Dies kann die Komplexität reduzieren und die Verwaltung erleichtern, da alle Komponenten als eine Einheit entwickelt, getestet und bereitgestellt werden.

Im Kontext der vorliegenden Anwendung würde dies bedeuten, dass sowohl die Verarbeitung der Wetterdaten als auch die geospatialen Analysen in einer einzigen Anwendung zusammengefasst wären. Dies könnte von Vorteil sein, da alle Informationen zur Bewertung eines Standorts auf einmal gesammelt werden, was die Integrationsphase vereinfacht.

Allerdings bringt eine monolithische Architektur erhebliche Nachteile mit sich. Die Flexibilität und Skalierbarkeit wären stark eingeschränkt, da die gesamte Anwendung als Einheit skaliert werden müsste. Bei einem Anstieg der Nutzerzahlen, beispielsweise wenn viele Grundstückseigentümer gleichzeitig potenzielle Standorte abfragen, wäre es nicht möglich, nur die wetterbezogenen Funktionen zu skalieren. Zudem würde die Entwicklung und Wartung der Anwendung erschwert, da keine einzelnen Teams unabhängig an spezifischen Services arbeiten könnten. Änderungen an einer Funktion könnten unbeabsichtigte Auswirkungen auf andere Teile der Anwendung haben.

\subsection{On-Premise-Lösungen}

On-Premise-Lösungen bieten der vorliegenden Anwendung mehrere Vorteile, insbesondere in Bezug auf Souveränität und Sicherheit.

\textbf{Kontrolle über die Infrastruktur:} On-Premise-Lösungen ermöglichen eine vollständige Kontrolle über die Infrastruktur und die Daten. Im Kontext der Anwendung könnte eine traditionelle serverbasierte Lösung alle Funktionen auf einem zentralen Server bündeln. Dies würde den initialen Aufwand für das Hosting und die Verwaltung der Anwendung verringern, da alle Wetter- und Geoinformationen direkt von einem Server abgerufen werden könnten.

\textbf{Souveränität und Sicherheit:} Durch den Betrieb der Anwendung auf eigenen Servern wird die Sicherheit der Daten erhöht, da sensible Informationen innerhalb der eigenen Räumlichkeiten gespeichert werden. Dies verringert die Risiken, die mit externen Cloud-Diensten verbunden sind, und ist besonders wichtig, wenn es um den Schutz kritischer Daten geht. In der aktuellen Anwendung hat die DSGVO zwar keine hohe Relevanz, da keine personenbezogenen Daten verarbeitet werden, jedoch könnten zukünftige Erweiterungen, die ein Pricing-Modell implementieren und somit sensible Informationen wie Zahlungsdaten erfordern, die Notwendigkeit für On-Premise-Lösungen verstärken.

Dennoch bringt eine monolithische Architektur, wie sie in On-Premise-Lösungen häufig anzutreffen ist, einige Herausforderungen mit sich. Die Agilität und Flexibilität der Anwendung wären eingeschränkt, da alle Funktionen als Einheit betrieben werden müssten. Im Falle eines Anstiegs der Nutzerzahlen könnte die Anwendung nicht effizient skaliert werden, was die Reaktionsfähigkeit und Leistung der Anwendung beeinträchtigen würde.