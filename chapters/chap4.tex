\chapter{Datensicherheit und Datenschutz in Cloud-Native-Anwendungen}
Im Kontext der vorliegenden Anwendung hat die Datenschutz-Grundverordnung (DSGVO) derzeit eine geringe Relevanz, da keine personenbezogenen Daten verarbeitet werden. Nutzer benötigen keinen Account, um die Anwendung zur Suche nach Standorten innerhalb Deutschlands zu verwenden, weshalb auch keine entsprechenden Datenschutzmechanismen implementiert wurden.

Für zukünftige Versionen der Anwendung ist jedoch ein Pricing-Modell vorgesehen, welches den Zugriff auf Standorte außerhalb Europas ermöglicht. Dieses Modell würde die Erfassung personenbezogener Daten, wie Name und Zahlungsinformationen, erfordern sowie die Einwilligung der Nutzer zur Datenverarbeitung.

Die Sicherheit dieser Daten muss während der Erzeugung, Übertragung, Nutzung, Speicherung, Archivierung und Löschung gewährleistet sein. Entsprechende Sicherheitsmaßnahmen müssen sowohl auf Netzwerk- als auch auf Anwendungsebene implementiert werden. Dabei sind wichtige Aspekte zu berücksichtigen, wie beispielsweise der Standort der Server, auf denen die Nutzerdaten gespeichert werden.